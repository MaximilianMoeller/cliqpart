% !TEX TS-program = lualatex
\documentclass{beamer}

% !TEX TS-program = lualatex

% Preamble für Uni-Mitschriften
% Stark inspiriert von https://github.com/gillescastel/university-setup/blob/master/preamble.tex
%  ┌────────┐
%  │ Basics │
%  └────────┘
% westeuropäische Codierung (verbessert Trennung und copy/paste)
\usepackage[T1]{fontenc}
\usepackage{fontspec}
% spezielle Sonderzeichen
\usepackage{textcomp}
% deutsches Sprachpaket
\usepackage[ngerman, english]{babel}
\usepackage{url}
% Einbindung von Bildern
\usepackage{graphicx}
% Bilder auf der Titelseite
\usepackage{titlepic}
% verbesserte floating-Objekte (z.B. figures/tables)
\usepackage{float}
% verbesserter Schriftsatz (nicht sichtbare Abweichungen)
\usepackage[]{microtype}
% verbesserte Tabellen
\usepackage{booktabs}
% Kontrolle über enumerate, itemize und description
\usepackage{enumitem}
% Platz zwischen Paragraphen
\usepackage{parskip}
% Keine Zeilennummern auf leeren Seiten
\usepackage{emptypage}
\usepackage{subcaption}
% Text in mehreren Spalten
\usepackage{multicol}
% Mehr Farben
\usepackage{xcolor}
% Lorem ipsum mit \blindtext
\usepackage[]{blindtext}
% Ersetzt etwas durch nichts (debugging)
\newcommand\hide[1]{}
% Zitate
\usepackage{csquotes}
\usepackage[style=alphabetic]{biblatex}
\addbibresource{../../references.bib}

% ┌────────────┐
% │ Mathematik │
% └────────────┘
% mathematische Pakete
\usepackage{amsmath, amsfonts, mathtools, amsthm, amssymb, mathrsfs}
% Terme streichen (optional mit Wert, z.B. Wert -> 0)
\usepackage{cancel}
% Bold math
\usepackage{bm}
% Shortcuts für Zahlenbereiche
\newcommand\N{\ensuremath{\mathbb{N}}}
\newcommand\Z{\ensuremath{\mathbb{Z}}}
\newcommand\Q{\ensuremath{\mathbb{Q}}}
\newcommand\R{\ensuremath{\mathbb{R}}}
\newcommand\C{\ensuremath{\mathbb{C}}}
% Funktionen und variablen aus mehreren Buchstaben
\newcommand{\var}[1]{\mathit{\text{#1}}}
\newcommand{\func}[1]{\operatorname{\text{#1}}}

% Mehr Sonderzeichen (z.B. Blitz oder eckige Doppelklammern)
\usepackage{stmaryrd}
% Fixt komische Schriftartfehler für stmaryrd
\SetSymbolFont{stmry}{bold}{U}{stmry}{m}{n}

% Widerspruch als Blitz
\newcommand\contra{\scalebox{1.5}{$\lightning$}}
% = mit "def" darüber
\newcommand\defeq{\stackrel{\text{def}}{=}}
% :<=>
\usepackage{colonequals}
\newcommand\logeq{\ratio\Longleftrightarrow}
% logical proof (inference rules)
\usepackage[]{proof}

% SI-Einheiten
\usepackage{siunitx}  
\sisetup{locale = DE}

% ┌──────────────┐
% │ Environments │
% └──────────────┘
% Kästen um Definitionen etc.
\usepackage{mdframed}
% Layout der Kästen
\mdfsetup{skipabove=1em,skipbelow=0em}
\theoremstyle{definition}
\mdfdefinestyle{theoremstyle}{
	linewidth=0.6pt,%
	frametitlerule=true,%
	frametitlerulewidth=0.4pt,%
	frametitlebackgroundcolor=gray!30,
	nobreak=true,
	}
\mdtheorem[style=theoremstyle]{definition}{Definition}[section]
\mdtheorem[style=theoremstyle]{theorem}[definition]{Theorem}
\mdtheorem[style=theoremstyle]{prop}[definition]{Proposition}
\mdtheorem[style=theoremstyle]{lemma}[definition]{Lemma}
\mdtheorem[style=theoremstyle]{corollary}[definition]{Corollary}
\mdtheorem[style=theoremstyle]{axiom}{Axiom}
\newtheorem*{example}{Example}
\surroundwithmdframed[innertopmargin=0, topline=false, bottomline=false, rightline=false, linewidth=4pt, linecolor=gray, backgroundcolor=lightgray]{example}
\newtheorem*{notation}{Notation}
\newtheorem*{terminology}{Terminology}
\newtheorem*{note}{Note}
\newtheorem*{problem}{Problem}

% deutsche Titel
\newmdenv[innertopmargin=0, topline=false, bottomline=false, rightline=false, linewidth=4pt, linecolor=gray, backgroundcolor=lightgray, frametitle={Beispiel:}]{beispiel}
\newmdenv[innertopmargin=0, topline=false, bottomline=false, rightline=false, linewidth=2pt, linecolor=gray, frametitle={Merke:}]{merke}
\mdtheorem[style=theoremstyle]{corollar}[definition]{Corollar}
\newtheorem*{terminologie}{Terminologie}

\newenvironment{subproof}[1][\proofname]{%
  \renewcommand{\qedsymbol}{$\blacksquare$}%
  \begin{proof}[#1]%
}{%
  \end{proof}%
}

\newenvironment{subsubproof}[1][\proofname]{%
  \renewcommand{\qedsymbol}{$\blacklozenge$}%
  \begin{proof}[#1]%
}{%
  \end{proof}%
}

% Fix some spacing
% http://tex.stackexchange.com/questions/22119/how-can-i-change-the-spacing-before-theorems-with-amsthm
\def\thm@space@setup{%
  \thm@preskip=\parskip \thm@postskip=0pt
}
\makeatother

% ┌──────┐
% │ tikz │
% └──────┘
\usepackage{tikz}
\usetikzlibrary{arrows, automata, positioning}
\tikzset{
	->, % directed edges
	>=stealth, % bold arrow heads
	node distance = 2cm,
	default/.style={circle, thin, draw=black, fill=gray!25}
}

% ┌────────────┐
% │ algorithms │
% └────────────┘
\usepackage{algorithm}
\usepackage[noend]{algpseudocode}

%  ┌───────────┐
%  │ sonstiges │
%  └───────────┘
\usepackage{hyperref}

\usepackage{tudscrcolor}

% Beamer styling
\usetheme{Dresden}
\setbeamertemplate{itemize items}[circle]
%\setbeameroption{hide notes} % Only slides
\setbeameroption{show notes on second screen=bottom} % notes and slides
\AtBeginSection[]{
  \begin{frame}
  \vfill
  \centering
  \begin{beamercolorbox}[sep=8pt,center,shadow=true,rounded=true]{title}
    \usebeamerfont{title}\insertsectionhead\par%
  \end{beamercolorbox}
  \vfill
  \end{frame}
}
\renewcommand{\arraystretch}{1.3}


\author{Maximilian Moeller}
\title{Evaluation of separation routines for some classes of inequalities of the Clique Partitioning polytope}
\subtitle{Colloquium “Großer Beleg”}
\date{21.11.2023}

\begin{document}
	
\frame{\titlepage}

\begin{frame}
\frametitle{Agenda}
\begin{enumerate}
	\item Introduction to Clique Partitioning
	\item Separating Inequalities
	\item Data
	\item Empirical Results
	\item Conclusion
\end{enumerate}
\end{frame}

\section{Introduction}
\begin{frame}
\frametitle{Clique Partitioning}
\begin{columns}
\begin{column}{0.5\textwidth}
\begin{itemize}
	\item clustering based on pairwise similarities
	\item framework for aggregation of binary relations
	\item useful in biology, medicine
\end{itemize}
	\note[item]{biology: taxonomy of animals (\eg whales and dolphins)}
	\note[item]{medicine: clustering of organoids in light microscopy images}
	\note[item]{probably also operations research}
\end{column}
\begin{column}{0.5\textwidth}  %%<--- here
	\centering
	\begin{tikzpicture}
        \def \n{8};
        \def \radius{1.5cm};
        \def \r{0.3cm};
        \foreach \i in {0,...,\n}{
			\node[vertex] (\i) at ({\i/(\n+1)*360+90}:\radius) {};
			\node at ({\i/(\n+1)*360+90}:\radius+\r) {$v_{\i}$};
        }
        \foreach \i in {0,...,\n}{
            \foreach \j in {\i,...,\n}{
                \draw[dotted, gray] (\i) -- (\j);
        	}
        }
        \foreach \i/\j in {1/8, 1/3, 1/5, 8/3, 8/5, 3/5} {
			\draw[line width=0.2mm] (\i) -- (\j);
        }
        \foreach \i/\j in {4/6, 4/7, 6/7} {
			\draw[line width=0.2mm] (\i) -- (\j);
        }
	\end{tikzpicture}
\end{column}
\end{columns}
\end{frame}

\begin{frame}
	\frametitle{Clique Partitioning}
	\begin{columns}
		\begin{column}{0.5\textwidth}
		\onslide<1->{
		\begin{definition}
			Given a graph $G=(V,E)$, a subset of edges $A \subseteq E$ is called a \textit{clique partitioning of $G$} if there exists a partition $\Gamma = \left\{ W_1, W_2, \ldots, W_k \right\}$ of $V$ such that
			\[
				A = \bigcup_{i=1}^k \left\{ (u,v) \in E \mid u,v \in W_i \right\}.
			\]
		\end{definition}}
		\note[item]{\textcolor{red}{show} which partition is induced}
		\note[item]{typically only complete graphs considered (indicates for every pair of nodes)}
		\note[item]{then one-to-one corresponds to an equivalence relation}
		\note[item]{induces complete subgraphs \textit{clique} partitioning}
		\end{column}
		\begin{column}{0.5\textwidth}  %%<--- here
			\centering
			\begin{tikzpicture}
			\onslide<1->{
		        \def \n{8};
		        \def \radius{1.5cm};
		        \def \r{0.3cm};
		        \foreach \i in {0,...,\n}{
					\node[vertex] (\i) at ({\i/(\n+1)*360+90}:\radius) {};
					\node at ({\i/(\n+1)*360+90}:\radius+\r) {$v_{\i}$};
		        }
		        \foreach \i in {0,...,\n}{
		            \foreach \j in {\i,...,\n}{
		                \draw[dotted, gray] (\i) -- (\j);
		        	}
			}}
			\only<2>{
		        \foreach \i/\j in {1/8, 1/3, 1/5, 8/3, 8/5, 3/5} {
					\draw[line width=0.2mm] (\i) -- (\j);
		        }
		        \foreach \i/\j in {4/6, 4/7, 6/7} {
					\draw[line width=0.2mm] (\i) -- (\j);
			}}
			\end{tikzpicture}
		\end{column}
	\end{columns}
\end{frame}

\begin{frame}
	\frametitle{The Clique Partitioning Problem}	
	Input: complete graph $K_{n} = (V_{n}, E_{n})$ and edge costs $w \in \R^{E_{n}}$

	Compute: a clique partitioning of minimum weight

	\begin{align*}
	\min && &\sum_{e \in E_{n}} w_{e} x_{e} \\[1.5ex]
	\text{s.t.} && &\forall e \in E_{n} \colon x_{e} \in \left\{ 0,1 \right\}  \\
				&& &x \text{ characterizes a clique partitioning} \\
	\end{align*}
	\note[item]{characterize clique partitionings by binary vectors}
	\note[item]{1 means it is in the clique partitioning and 0 means it is not in it}
	\note[item]{exactly opposite to \textcolor{red}{multicut} (equivalent problem)}
\end{frame}


\begin{frame}
	\frametitle{The Clique Partitioning Problem}	
	\begin{columns}
		\begin{column}{0.5\textwidth}
			\begin{itemize}
				\item<1-> \textsc{NP}-complete
				\item<2-> Trivial cases:
					\begin{itemize}
						\item<3-> $w \geq \bm{0}$	
						\item<4-> $w \leq \bm{0}$
						\item<5-> $w$ induces cliques
					\end{itemize}
			\end{itemize}
			\note[item]{blue drawn are edges with negative weights, all others non-negative}
			\note[item]{interesting instances need to have ‘conflicts’}
		\end{column}
		\begin{column}{0.5\textwidth}
			\onslide<3->{
			\begin{tikzpicture}
		        \def \n{8};
		        \def \radius{1.5cm};
		        \def \r{0.3cm};
		        \foreach \i in {0,...,\n}{
					\node[vertex] (\i) at ({\i/(\n+1)*360+90}:\radius) {};
					\node at ({\i/(\n+1)*360+90}:\radius+\r) {$v_{\i}$};
		        }
		        \foreach \i in {0,...,\n}{
		            \foreach \j in {\i,...,\n}{
						\only<3>{\draw[gray, dotted] (\i) -- (\j);}
						\only<5>{\draw[gray] (\i) -- (\j);}
		        	}
			}
			\only<4>{
		        \foreach \i in {0,...,\n}{
		            \foreach \j in {\i,...,\n}{
		                \draw (\i) -- (\j);
		    	}
			}}
			\only<5>{
		        \foreach \i/\j in {1/8, 1/3, 1/5, 8/3, 8/5, 3/5} {
					\draw[line width=0.4mm, color=cdblue] (\i) -- (\j);
		        }
		        \foreach \i/\j in {4/6, 4/7, 6/7} {
					\draw[line width=0.4mm, color=cdblue] (\i) -- (\j);
			}}}
			\end{tikzpicture}
		\end{column}
	\end{columns}
\end{frame}


\begin{frame}
	\frametitle{An ILP-formulation of Clique Partitioning}	
	\note[item]{we want to make use of (I)LP techniques: linear inequalities}
	characteristic vectors $x$ \onslide<2->{need to satisfy \textit{triangle inequalities} ($i,j,k \in V_{n}$, pairwise distinct)}
	\note[item]{explain support graph: left hand side can be drawn as graph, dotted negative, solid positive}
	\note[item]{triangle inequalities ensure transitivity}
	\note[item]{for the same $i,j,k$ there are actually three triangles (\textcolor{red}{show})}
\begin{align*}
	\onslide<2->{x_{ij} + x_{jk} - x_{ik} &\leq 1\\}
	\onslide<3->{x_{ij} - x_{jk} + x_{ik} &\leq 1\\}
	\onslide<3->{- x_{ij} + x_{jk} + x_{ik} &\leq 1\\}
\end{align*}
\vspace*{-2\baselineskip}
\onslide<2->{
\begin{figure}
	\centering
	\begin{subfigure}[b]{0.3\textwidth}
		\centering
			\begin{tikzpicture}
			    \def \n{2};
			    \def \radius{1.5cm};
			    \def \r{0.3cm};
			    \foreach \i/\j in {0/i,1/j,2/k}{
					\node[vertex] (\i) at ({\i/(\n+1)*360+90}:\radius) {};
					\node at ({\i/(\n+1)*360+90}:\radius+\r) {$v_{\j}$};
			    }
			    \draw (1) -- (2);
			    \draw[dashed] (2) -- (0);
			    \draw (1) -- (0);
			\end{tikzpicture}
	\end{subfigure}
	\hfill
	\only<3->{
	\begin{subfigure}[b]{0.3\textwidth}
		\centering
			\begin{tikzpicture}
			    \def \n{2};
			    \def \radius{1.5cm};
			    \def \r{0.3cm};
			    \foreach \i/\j in {0/i,1/j,2/k}{
					\node[vertex] (\i) at ({\i/(\n+1)*360+90}:\radius) {};
					\node at ({\i/(\n+1)*360+90}:\radius+\r) {$v_{\j}$};
			    }
			    \draw[dashed] (1) -- (2);
			    \draw (2) -- (0);
			    \draw (1) -- (0);
			\end{tikzpicture}
	\end{subfigure}
	\hfill
	\begin{subfigure}[b]{0.3\textwidth}
		\centering
			\begin{tikzpicture}
			    \def \n{2};
			    \def \radius{1.5cm};
			    \def \r{0.3cm};
			    \foreach \i/\j in {0/i,1/j,2/k}{
					\node[vertex] (\i) at ({\i/(\n+1)*360+90}:\radius) {};
					\node at ({\i/(\n+1)*360+90}:\radius+\r) {$v_{\j}$};
			    }
			    \draw (1) -- (2);
			    \draw (2) -- (0);
			    \draw[dashed] (1) -- (0);
			\end{tikzpicture}
	\end{subfigure}}
\end{figure}}
\end{frame}

\begin{frame}
\frametitle{The Task}
\note[item]{original task: branch-and-cut algorithm, \ie solving the ILP}
\note[item]{consists of multiple parts: \textbf{cutting planes for bounds}, finding good solutions and branch management}
\onslide<2->{
focus on cutting plane procedure:
\begin{itemize}
	\item<2-> solve LP relaxation of the problem
	\item<3-> tighten the solution iteratively
	\note[item]{explain overall iteration procedure}
	\item<4-> hopefully arrive at integral solution
	\item<5-> ???
	\item<5-> profit
\end{itemize}}
\note[item]{even if not integral, a few percent off might be good for practical use}
\note[item]{however, no guarantee to arrive at integral or within a constant factor}
\end{frame}

\section{Separating Inequalities}


\begin{frame}
	\frametitle{Ingredients}	
	\begin{itemize}
		\item<1-> classes of facet-defining linear inequalities
		\item<2-> algorithms for separating them
		\item<3-> data to test on (next section)
	\end{itemize}
	\note[item]{explain: what is separation (check (and produce))}
	\note[item]{explain: facet-defining}
\end{frame}


\begin{frame}
	\frametitle{Triangle Inequalities (\texttt{Δ})}	
	Using complete enumeration $\mathcal{O}(n^{3})$
	\note[item]{Separation algorithms: separators}
	\note[item]{separators get abbreviations}

	Four variations
	\vspace*{2\baselineskip}
\onslide<2>{
\begin{table}
	\begin{tabular}{@{}lcc@{}}
		\toprule
		abbreviation & \texttt{maxcut} & \texttt{var\_once} \\
		\midrule
		\texttt{Δ} & 400 & – \\ 
		$\texttt{Δ}_{\infty}$ & $\infty$ & – \\
		$\texttt{Δ}^{\leq 1}$ & 400 & \checkmark \\
		$\texttt{Δ}_{\infty}^{\leq 1}$ & $\infty$ & \checkmark \\
		\bottomrule
	\end{tabular}
	
	\note[item]{\texttt{maxcut} is for limiting lp size}
	\note[item]{\texttt{var\_once} is for solving the graph evenly}
\end{table}}
\end{frame}

\begin{frame}
	\frametitle{2-Partition Inequalities (\texttt{st})}	
	\note[item]{Generalization of triangle inequalities}
	Let $S, T \subseteq V_{n} $ disjoint.

	2-\textit{partition inequality} \quad ($\left[ S,T \right]$-inequality)
	\[
		\sum_{s \in S}^{} \sum_{t \in T}^{} x_{st} - \sum_{\substack{s, s' \in S \\ s \neq s'}}^{} x_{ss'} - \sum_{\substack{t, t' \in T \\ t \neq t'}}^{} x_{tt'} \leq \func{min}( \lvert S \rvert, \lvert T \rvert )
	\]
	\vspace*{-2\baselineskip}
	\onslide<2->{
	\begin{columns}
				\begin{column}{0.5\textwidth}
					facet defining for $\lvert S \rvert \neq \lvert  T \rvert$
				\end{column}
				\begin{column}{0.5\textwidth}
			\begin{tikzpicture}
        \def \n{4};
        \def \radius{1.5cm};
        \def \r{0.3cm};
        \foreach \i in {0,...,\n}{
			\node[vertex] (\i) at ({\i/(\n+1)*360+90}:\radius) {};
			\node at ({\i/(\n+1)*360+90}:\radius+\r) {$v_{\i}$};
        }
        \foreach \i/\j in {0/1, 1/2, 0/2, 3/4} {
			\draw[dashed, line width=0.35mm] (\i) -- (\j);
        }
        \foreach \i/\j in {0/3, 0/4, 1/3, 1/4, 2/3, 2/4} {
			\draw (\i) -- (\j);
        }
	\end{tikzpicture}}
		\end{column}
	\end{columns}
\end{frame}


\begin{frame}
	\frametitle{Separation of 2-Partition Inequalities (\texttt{st})}	
	\begin{itemize}
		\item<1-> already \textsc{NP}-hard for any fixed $\lvert S \rvert$
		\item<2-> using two heuristics: $\texttt{st}^{1}$ and $\texttt{st}^{2}$
		\item<3-> only search for constraints with $\lvert S \rvert = 1$
	\end{itemize}
	\note[item]{no use in using approximation if it is np-hard again}
	\note[item]{not explaining details, involve random choosing of an order of some nodes. Thus re-run three times.}
\end{frame}

\section{Data}
\begin{frame}
	\frametitle{Natural instances}	
	\begin{itemize}
		\item<1-> \texttt{cetacea} (30)
		\item<1-> \texttt{cetacea} (36)
		\item<2-> \texttt{football} (115)
		\item<2-> \texttt{adjnoun} (112)
		\item<2-> \texttt{polbooks} (105)
		\item<2-> \texttt{lesmis} (77)
		\item<2-> \texttt{dolphins} (62)
		\item<2-> \texttt{karate} (34)
		\item<3-> \texttt{organoid\_[size]\_[difficulty]} (160, 100, 80, 40)
	\end{itemize}
\end{frame}

\begin{frame}
	\frametitle{Random instances}	
	\begin{itemize}
		\item<1-> Binary \texttt{r\_binary\_[size]} \\
			$\texttt{size} \in \left\{ 25, 50 \right\}$
		\item<2-> Uniform \texttt{r\_uniform\_[size]\_[lb]\_[ub]} \\
			$\texttt{size} \in \left\{ 25, 50 \right\}$ \\
			$(\texttt{lb}, \texttt{ub}) \in \left\{ (-10, 10), (-100, 100), (-10, 100) \right\}$
		\item<3-> Normal \texttt{r\_normal\_[size]\_[mu]\_[sigma]} \\
			$\texttt{size} \in \left\{ 25, 50 \right\}$ \\
			$(\text{\texttt{mu}}, \text{\texttt{sigma}}) \in \left\{ 0, 0.5, 2 \right\} \times \left\{ 0.5, 1, 2 \right\}$

	\end{itemize}
	\end{frame}

\section{Empirical Results}

\section{Conclusion}


\end{document}
