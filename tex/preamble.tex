% !TEX TS-program = lualatex

% Preamble für Uni-Mitschriften
% Stark inspiriert von https://github.com/gillescastel/university-setup/blob/master/preamble.tex
%  ┌────────┐
%  │ Basics │
%  └────────┘
%
\usepackage[TU]{fontenc} %to enable bold small-caps letters

\usepackage{fontspec}
\setsansfont{Latin Modern Roman}
% spezielle Sonderzeichen
\usepackage{textcomp}
% deutsches Sprachpaket
\usepackage[ngerman, english]{babel}
%\usepackage{url}
% Einbindung von Bildern
\usepackage{graphicx}
% Bilder auf der Titelseite
\usepackage{titlepic}
% verbesserte floating-Objekte (z.B. figures/tables)
\usepackage{float}
% verbesserter Schriftsatz (nicht sichtbare Abweichungen)
\usepackage[]{microtype}
% kann erkennen, ob ein space gebraucht wird oder nicht (praktisch für newcommands)
\usepackage{xspace}
% verbesserte Tabellen
\usepackage{booktabs}
% Kontrolle über enumerate, itemize und description
\usepackage{enumitem}
% Platz zwischen Paragraphen
\usepackage{parskip}
% Keine Zeilennummern auf leeren Seiten
\usepackage{emptypage}
\usepackage{subcaption}
% Text in mehreren Spalten
\usepackage{multicol}
% Mehr Farben
\usepackage[dvipsnames]{xcolor}
% Lorem ipsum mit \blindtext
\usepackage[]{blindtext}
% Todos
\usepackage[colorinlistoftodos]{todonotes}
% Ersetzt etwas durch nichts (debugging)
\newcommand\hide[1]{}
% Zitate
\usepackage{csquotes}
% Bibliography
\usepackage[backend=biber, style=alphabetic]{biblatex}
\addbibresource{references.bib}

% e.g. and i.e.
\newcommand\eg{\textit{e}.\textit{g}.\ }
\newcommand\ie{\textit{i}.\textit{e}.\ }

% ┌────────────┐
% │ Mathematik │
% └────────────┘
% mathematische Pakete
\usepackage{amsmath, amsfonts, mathtools, amsthm, amssymb, mathrsfs}
% Terme streichen (optional mit Wert, z.B. Wert -> 0)
\usepackage{cancel}
% Bold math
\usepackage{bm}
% Shortcuts für Zahlenbereiche
\newcommand\N{\ensuremath{\mathbb{N}}}
\newcommand\Z{\ensuremath{\mathbb{Z}}}
\newcommand\Q{\ensuremath{\mathbb{Q}}}
\newcommand\R{\ensuremath{\mathbb{R}}}
\newcommand\C{\ensuremath{\mathbb{C}}}
% Funktionen und variablen aus mehreren Buchstaben
\newcommand{\var}[1]{\mathit{\text{#1}}}
\newcommand{\func}[1]{\operatorname{\text{#1}}}

% Mehr Sonderzeichen (z.B. Blitz oder eckige Doppelklammern)
\usepackage{stmaryrd}
% Fixt komische Schriftartfehler für stmaryrd
\SetSymbolFont{stmry}{bold}{U}{stmry}{m}{n}

% Widerspruch als Blitz
\newcommand\contra{\scalebox{1.5}{$\lightning$}}
% = mit "def" darüber
\newcommand\defeq{\stackrel{\text{def}}{=}}
% :<=>
\usepackage{colonequals}
\newcommand\logeq{\ratio\Longleftrightarrow}
% logical proof (inference rules)
\usepackage[]{proof}

% SI-Einheiten
%\usepackage{siunitx}  
%\sisetup{locale = DE}

% ┌──────────────┐
% │ Environments │
% └──────────────┘
% Kästen um Definitionen etc.
\usepackage{mdframed}
% Layout der Kästen
\mdfsetup{skipabove=1em,skipbelow=0em}
\theoremstyle{definition}
\mdfdefinestyle{theoremstyle}{
	linewidth=0.6pt,%
	frametitlerule=true,%
	frametitlerulewidth=0.4pt,%
	frametitlebackgroundcolor=gray!30,
	nobreak=true,
	}
\mdtheorem[style=theoremstyle]{definition}{Definition}[section]
\mdtheorem[style=theoremstyle]{theorem}[definition]{Theorem}
\mdtheorem[style=theoremstyle]{prop}[definition]{Proposition}
\mdtheorem[style=theoremstyle]{lemma}[definition]{Lemma}
\mdtheorem[style=theoremstyle]{corollary}[definition]{Corollary}
\mdtheorem[style=theoremstyle]{axiom}{Axiom}
\newtheorem*{example}{Example}
\surroundwithmdframed[innertopmargin=0, topline=false, bottomline=false, rightline=false, linewidth=4pt, linecolor=gray, backgroundcolor=lightgray]{example}
\newtheorem*{notation}{Notation}
\newtheorem*{terminology}{Terminology}
\newtheorem*{note}{Note}
\newtheorem*{problem}{Problem}

% deutsche Titel
\newmdenv[innertopmargin=0, topline=false, bottomline=false, rightline=false, linewidth=4pt, linecolor=gray, backgroundcolor=lightgray, frametitle={Beispiel:}]{beispiel}
\newmdenv[innertopmargin=0, topline=false, bottomline=false, rightline=false, linewidth=2pt, linecolor=gray, frametitle={Merke:}]{merke}
\mdtheorem[style=theoremstyle]{corollar}[definition]{Corollar}
\newtheorem*{terminologie}{Terminologie}

\newenvironment{subproof}[1][\proofname]{%
  \renewcommand{\qedsymbol}{$\blacksquare$}%
  \begin{proof}[#1]%
}{%
  \end{proof}%
}

\newenvironment{subsubproof}[1][\proofname]{%
  \renewcommand{\qedsymbol}{$\blacklozenge$}%
  \begin{proof}[#1]%
}{%
  \end{proof}%
}

% Fix some spacing
% http://tex.stackexchange.com/questions/22119/how-can-i-change-the-spacing-before-theorems-with-amsthm
\def\thm@space@setup{%
  \thm@preskip=\parskip \thm@postskip=0pt
}
\makeatother

% ┌──────┐
% │ tikz │
% └──────┘
\usepackage{tikz}
\usetikzlibrary{arrows, automata, positioning}
%\tikzset{
%	->, % directed edges
%	>=stealth, % bold arrow heads
%	node distance = 2cm,
%	default/.style={circle, thin, draw=black, fill=gray!25}
%}
\tikzstyle{vertex}=[circle, draw, fill=white, inner sep=0pt, minimum width=1ex]

% ┌────────────┐
% │ algorithms │
% └────────────┘
\usepackage{algorithm}
\usepackage[noend]{algpseudocode}

%  ┌───────────┐
%  │ sonstiges │
%  └───────────┘
\usepackage{hyperref}
\usepackage[capitalise]{cleveref}
