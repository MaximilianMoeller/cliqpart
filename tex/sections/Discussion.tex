\section{Conclusion}\label{sec:discussion}
\CP is a computationally difficult optimization problem.
We achieved good results with respect to running time and relative gap to optimal soultions on most natural instances,
while random instances posed a much harder challenge.
This is due to the fact that in natural instances there is often some underlying structure to the problem of clustering.
If, \eg in the \texttt{cats} instance, cat $a$ and cat $b$ are similar wrt.\ to some characteristics and then also cat $b$ and cat $c$ are similar,
it is very likely that cat $a$ and $c$ are also similar.
There are no such underlying sturctures in randomly created instances, which makes the problem much harder in general.

We confirm the impression of \cite{grotschelCuttingPlaneAlgorithm1989} that separating triangle- and $\left[ S,T \right]$-inequalities is often enough to arrive at integral and thus optimal solutions.
Especially for practical applications, the separation of half- and two-chorded odd cycle inequalities does not seem to be worth the additional computational resources required.
