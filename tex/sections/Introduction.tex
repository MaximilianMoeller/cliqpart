\section{Introduction}
The goal of this project was to gain some familiarity with the \CP problem in particular and combinatorial optimization techniques in general.
Especially knowledge about classes of facet-defining inequalities is of vital interest for solving combinatorial optimization problems through branch-and-cut algorithms.
Therefore, the task of the project was to review the available literature on such classes and implement separation routines for as many of them as possible.
The relatively new class of $2$-chorded-half-cycle inequalities first described by \cite{andresPolyhedralStudyLifted2022} was of particular interest, because their practical use was unknown so far.
The separation routines should then be tested on data available from the literature as well as new instances.

\subsection{The \CP problem}\label{subsec:cp}
The \CP problem is a combinatorial optimization problem that was first investigated by \cite{grotschelFacetsCliquePartitioning1990}.
The objective is to cluster a given set of objects with respect to their pairwise similarities into ‘maximally homogenous’ sets.

\begin{definition}[\CP]\label{def:cp}
	Given a graph $G=(V,E)$, a subset of edges $A \subseteq E$ is called a \textit{clique partitioning of $G$} if there exists a partition $\Gamma = \left\{ W_1, W_2, \ldots, W_k \right\}$ of $V$ such that
	\[
		A = \bigcup_{i=1}^k \left\{ (u,v) \in E \mid u,v \in W_i \right\}.
	\]
We denote the set of all clique partitionings of a graph $G$ by $\func{CP}(G)$.
\end{definition}
\begin{note}
A clique partitioning $A$ consists precisely of the edges that connect vertices within the same block $W_i$ of the partition $\Gamma$.
Thus, every clique partitioning $A$ induces an equivalence relation $\sim_A$ on $V$, and every equivalence relation on $V$ corresponds uniquely to a clique partitioning of $G$.
Typically, one only considers complete graphs $G = K_n$, as this allows for any two objects to end up in the same block of the partition.
In this case every block $W_i$ induces a clique, \ie a complete subgraph of $G$, hence the name \textit{clique} partitioning.
\end{note}

The \CP problem is then the following.
Given a complete graph $K_n=(V_n,E_n)$ and edge weights $w_e \in \R^{E_n}$, find a clique partitioning of minimal weight.
The problem is \textsc{NP}-hard in the strong sense \todo{add reference to wakabayashis phd thesis}.

\todo{ILP-formulation beschreiben}


Many variations of the problem are well studied, e.g. restricting the cardinality or the number (or both) of the obtained partitions.\todo{add references to original papers}
Partitioning possibly incomplete graphs is also studied under the name “\textsc{Multicut} problem”.
Here, one typically thinks of the edges that are ‘cut’, \ie the edges between blocks of the partitiong instead of within a block.
Hence the characteristic vectors and the inequalities for the \textsc{Multicut} problem are obtained by replacing a variable $x$ from the \CP problem with $(1-x)$.
Therefore, these two problems are equivalent for complete graphs in the sense that an optimal solution to one of them also yields an optimal solution to the other.
