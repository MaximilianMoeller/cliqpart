\section{Introduction}
\subsection{About this project}\label{subsec:about}
The goal of this student research project was to gain some familiarity with the \CP problem in particular and combinatorial optimization techniques in general.
Knowledge about classes of facet-defining inequalities is of vital interest for solving combinatorial optimization problems through branch-and-cut algorithms.
Therefore, the task of the project was to review the available literature on such classes and implement separation routines for as many of them as possible.
The relatively new class of \textit{half-chorded odd cycle inequalities} first described by \cite{andresPolyhedralStudyLifted2022} was of particular interest, because their practical use was unknown so far.
The separation routines should then be tested on data available from the literature as well as new instances.

The remainder of this section will give a brief overview of the \CP problem, but for an in-depth introduction we refer the reader to \cite{grotschelFacetsCliquePartitioning1990}.
The structure of the document is as follows:
\begin{table}[h]
	\centering
	\begin{tabular}{@{}ll@{}}
		\toprule
		Section & Topic \\
		\midrule
		\cref{sec:separation}& Separation routines for some classes of inequalities \\
		\cref{sec:data}& Instances used for computational experiments \\
		\cref{sec:experiments}& Setup, overall procedure and configuration of the algorithm \\
		\cref{sec:empirical_results}& Empirical results from our experiments \\
		\cref{sec:discussion}& Conclusion \\
		\bottomrule
	\end{tabular}
	\vspace{10pt}
	\caption{Structure of this document}
\end{table}

\subsection{The \CP problem}\label{subsec:cp_problem}
The \CP problem is a combinatorial optimization problem that was first investigated by \cite{grotschelFacetsCliquePartitioning1990} as a framework for aggregation of binary relations.
The objective is to cluster a given set of objects with respect to their pairwise similarities into ‘maximally homogenous’ sets.

\begin{definition}[\CP]\label{def:cp}
	Given a graph $G=(V,E)$, a subset of edges $A \subseteq E$ is called a \textit{clique partitioning of $G$} if there exists a partition $\Gamma = \left\{ W_1, W_2, \ldots, W_k \right\}$ of $V$ such that
	\[
		A = \bigcup_{i=1}^k \left\{ (u,v) \in E \mid u,v \in W_i \right\}.
	\]
\end{definition}
A clique partitioning $A$ consists precisely of the edges that connect vertices within the same block $W_i$ of the partition $\Gamma$.
Thus, every clique partitioning $A$ induces an equivalence relation $\sim_A$ on $V$.
\cref{fig:example_clique_partitioning} shows an example of a clique partitioning.

\begin{figure}[H]
	\centering
	\begin{tikzpicture}
        \def \n{8};
        \def \radius{1.5cm};
        \def \r{0.3cm};
        \foreach \i in {0,...,\n}{
			\node[vertex] (\i) at ({\i/(\n+1)*360+90}:\radius) {};
			\node at ({\i/(\n+1)*360+90}:\radius+\r) {$v_{\i}$};
        }
        \foreach \i in {0,...,\n}{
            \foreach \j in {\i,...,\n}{
                \draw[dotted, gray] (\i) -- (\j);
        	}
        }
        \foreach \i/\j in {1/8, 1/3, 1/5, 8/3, 8/5, 3/5} {
			\draw[line width=0.2mm] (\i) -- (\j);
        }
        \foreach \i/\j in {4/6, 4/7, 6/7} {
			\draw[line width=0.2mm] (\i) -- (\j);
        }
	\end{tikzpicture}
	\caption[Example clique partitioning]{An example of a clique partitioning $A$ for $K_{9}$. 
		Solid lines represent edges that are part of the clique partitioning and dotted lines represent edges $e \in E_{9} \setminus A$.
	The induced partition is $\Gamma = \left\{ \left\{ v_{0} \right\}, \left\{ v_{2} \right\}, \left\{ v_{1}, v_{3}, v_{5}, v_{8} \right\}, \left\{ v_{4}, v_{6}, v_{7} \right\} \right\}$.}
	\label{fig:example_clique_partitioning}
\end{figure}

Typically, one only considers complete graphs $K_n$, because then a clique partitioning indicates for every pair of nodes whether they are in the same block or not.
That is because for a complete graph $K_{n} = (V_{n}, \binom{V_{n}}{2})$, every equivalence relation on $V_{n}$ corresponds uniquely to a clique partitioning of $K_{n}$.
Consider on the other hand \cref{fig:incomplete_graph}.
We could not distinguish the partionings $\Gamma_{1} = \left\{ \left\{ i \right\}, \left\{ j \right\}, \left\{ k \right\} \right\}$ and $\Gamma_{2} = \left\{ \left\{ i, k \right\}, \left\{ j \right\} \right\}$, because they both correspond to the empty clique partitioning $A = \emptyset$.
However, in many applications of \CP (\eg the instances given in \cite{grotschelCuttingPlaneAlgorithm1989}), every partitioning should be a feasible solution.
We will therefore restrict our attention to the case of complete graphs for the rest of the paper.
With this assumption every block $W_i$ induces a clique, \ie a complete subgraph of $G$, hence the name \textit{clique} partitioning.

\begin{figure}[h]
	\centering
	\begin{tikzpicture}
		\node[vertex, label=below:{$i$}] (i) at (0,0) {};
		\node[vertex, label=below:{$j$}] (j) at (1,0) {};
		\node[vertex, label=below:{$k$}] (k) at (2,0) {};

		\draw (i) -- (j);
		\draw (j) -- (k);
	\end{tikzpicture}
	\caption{An example of an incomplete graph.}
	\label{fig:incomplete_graph}
\end{figure}

The \CP problem is then the following.
Given a complete graph $K_n=(V_n,E_n)$ and edge weights $w \in \R^{E_n}$, find a clique partitioning of minimal weight.
The problem is known to be \textsc{NP}-hard in the strong sense \cite{wakabayashiAggregationBinaryRelations1986}.
We will just briefly mention the trivial cases.
If all entries in $w$ are non-negative, the empty clique partitioning is one optimal solution.
Likewise, if all entries are non-positive, the maximal clique partitioning $A = E$ is an optimal solution.
Furthermore, if the non-positive edges in $w$ induce cliques, then the optimal partition will consists exactly of these clique.
More ‘interesting’ instances must therefore contain ‘conflicts’, \ie they must contain vertices adjacent to both positive and negative edges.

\subsection{An ILP-formulation of \CP}\label{subsec:ilp_formulation}
Any valid clique partitioning of a complete graph $K_{n}=(V_{n}, E_{n})$ can be characterized by a vector $x \in \left\{ 0,1 \right\}^{E_{n}}$.
For any edge $e \in E_{n}$ the value $x_{e} = 1$ means that $e \in A$, and $x_{e} = 0$ means that $e \notin A$.
The \CP polytope $\mathscr{P}_{n}$ is then the convex hull of all characteristic vectors of clique partitionings of the complete graph $K_{n}$.

Since the problem is $\var{NP}$-hard, a full description of this polytope by linear inequalities is unlikely, unless $\var{P} = \var{NP}$.
However, many classes of valid and facet-defining inequalities have been found by the research community.
Section 2 of \cite{andresPolyhedralStudyLifted2022} gives a comprehensive overview over known classes as well as over the history of the problem.

\cite{grotschelFacetsCliquePartitioning1990} showed that a vector $x \in \left\{ 0,1 \right\}^{E_{n}}$ characterizes a clique partitioning iff it satisfies the \textit{triangle inequalities}:
\begin{equation}\label{eq:triangle_inequality}
	 x_{ij} + x_{jk} - x_{ik} \leq 1
\end{equation}
for all pairwise distinct $i,j,k \in V_{n}$.
This leads us to the following ILP formulation of \CP.
\begin{align*}
	\min && &\sum_{e \in E_{n}} w_{e} x_{e} \\[1.5ex]
	\text{subject to } && &x_{e} \in \left\{ 0,1 \right\} && \forall e \in E_{n} \\
					   && &x_{uv} + x_{vw} - x_{uw} \leq 1 && \forall u,v,w \in V, \text{ pairwise distinct} \\
\end{align*}

The purpose of the triangle inequalities is to enforce transitivity of the induced equivalence relation.
Notice that for each set $\left\{ i,j,k \right\} \subseteq V$ of pairwise distinct vertices there are actually three triangle inequalities:
\begin{align*}
	x_{ij} + x_{jk} - x_{ik} &\leq 1\\
	x_{ij} - x_{jk} + x_{ik} &\leq 1\\
	- x_{ij} + x_{jk} + x_{ik} &\leq 1\\
\end{align*}
We will however refer to these three as \textit{the} triangle inequality with respect to $\left\{ i,j,k \right\}$.

In combination with an ILP solver (we used Gurobi \cite{gurobioptimizationllcGurobiOptimizerReference2023}) the triangle inequalities already suffice to arrive at optimal solutions.
However these computations might require great amounts of computational resources.
We observed that for some – admittedly ‘difficult’ – instances the solver could not produce a single solution within half an hour on fairly modern hardware (see \cref{subsec:experiments_setup}).

We will therefore follow \cite{grotschelFacetsCliquePartitioning1990} in using a cutting plane procedure, \ie solving LP relaxations of the problem and adding violated constraints iteratively to tighten the relaxation.
This method is neither guaranteed to produce integral solutions nor can it give an approximation to the optimal objective value within any fixed factor.
Nevertheless, one can imagine applications where a ‘good’ estimate on the optimal solution can already be of use.
We will compare the results of this approach with optimal solutions obtained by the ILP solver when available.
For bigger instances, comparing only the results obtained by different parametrizations of the cutting plane procedure might already give a ‘good’ estimate on the optimal objective value.

\subsection{The Multicut Problem}
We briefly mention that the \CP problem is equivalent to the multicut problem (see \eg \cite{chopraPartitionProblem1993}, \cite{dezaCliqueWebFacetsMulticut1992}).
The main difference is that for some clique partitioning $A$ of $K_{n}$, the corresponding multicut consist precisely of the edges $E_{n} \setminus A$.
For any edge $e \in E_{n}$ the value $x_{e} = 1$ for the multicut variable means that $e$ is part of the multicut, \ie not part of the correpsonding \CP, and vice versa.
We therefore get the following relation between the \CP polytope $\mathscr{P}_{n}$ and the multicut polytope $\mathscr{M}_{n}$:
\[
	\mathscr{M}_{n} = \left\{ \mathbbm{1}^{E_{n}} - x \mid x \in \mathscr{P}_{n} \right\}
\]
where $\mathbbm{1}^{E_{n}}$ denotes the vector of dimension $\lvert E_{n} \rvert$ with all entries equal to $1$.

Furthermore, the multicut problem is seldom restricted to the case of complete graphs.
However, any results obtained for the multicut problem of the complete graph can be easily transferred to the \CP problem and vice versa.
