\appendix
\begin{landscape}

\section{Empirical Data}\label{sec:empirical_data}

\newcommand{\mycsvtable}[1]{
	\resizebox{\linewidth}{!}{
		\csvreader[
			separator=semicolon,
			no head,
			respect sharp,
			respect percent,
			column count = 20,
			tabular={@{}rlccccccccccccccccccc@{}},
			table head = {\toprule
				&&&&&&\multicolumn{3}{c}{\texttt{Δ-st-1}}
				&\multicolumn{3}{c}{\texttt{Δ-st-2}}
				&\multicolumn{3}{c}{\texttt{Δ-st-12}}
				&&&&\multicolumn{3}{c}{\texttt{all}}
				\\ \cmidrule(lr){7-9}\cmidrule(lr){10-12}\cmidrule(lr){13-15}\cmidrule(lr){19-21}},
			table foot = {\bottomrule},
			late after first line=\\\midrule,
		]{#1}{}{\thecsvrow &\csvlinetotablerow}
	}
}

\begin{table}[H]
\centering
\caption{Computational results for \texttt{cetacea}}
\bigskip
\mycsvtable{analysisCSVs/cetacea_analysis.csv}
\label{tab:cetacea}
\end{table}

\cref{tab:cetacea} shows the empirical results for the \texttt{cetacea} data set.
The table is organized as follows:
\begin{itemize}
	\item 1: Each column contains statistics for one \textit{run configuration} (cf.\ \cref{subsec:run_configurations}),
		and run configurations containing at least one \texttt{st}-separator are further split to show results for the three repetitions of the experiment.
	\item 2-3: Row 2 shows the number of iterations it took for the run configuration to finish. 
		In bold is the least amount of iterations.
		Row 3 shows how often non-triangle-separators were called, which is at least as high as the number of iterations in which a non-triangle-separator was called (not shown).
	\item 4-9: Information about cutting planes. 
		Row 5 shows how many were added in total during the whole run (least amount in bold),
		row 6 and 7 show the maximum (resp. minimum) amount of cuts added per iteration.
		If the algorithm terminated because no violated constraint was found in the last iteration, it is excluded from row 7, because the minimum would then always be 0.
		Row 8 shows the total number of non-triangle-constraints added
		and row 9 shows the total number of constraints that were later removed from the LP.
	\item 10-12: Information about the obtained objective.
		Row 11 shows the objective value of the last iteration, \ie the tightest relaxation that this run configuration could produce.
		Row 12 shows the relative gap (cf.\ \cref{subsec:gaps}) to the optimal solution, and a “–” if no optimal solution is known.
		The tightest relaxations are in bold and row 12 indicates by a $^{*}$ when an integral solution was found.
	\item 13-16: Information about running time.
		Row 14 shows the maximum between the total running time of the algorithm in seconds and one millisecond to emphasize that durations beneath this threshold were not measured.
		Row 15 shows this time normalized to the running time of the run configuration \texttt{Δ}.
		Row 16 shows the total amount of time that the run configuration spent solving the LP. Again, values beneath one millisecond are shown as one millisecond.
\end{itemize}
All non-integer values (rows 10-16) are rounded to preserve at least some clarity.
It might thus happen, \eg, that a very small (but non-zero) relative gap is shown as $0.0\%$, but neither in bold nor with a $^{*}$.

\newpage

% grötschel wakabayashi
\begin{table}[p]
\centering
\caption{Computational results for \texttt{cats}}
\bigskip
\mycsvtable{analysisCSVs/wild_cats_analysis.csv}
\label{tab:cats}
\\[10pt]
The structure of the table is as for \cref{tab:cetacea}.
\end{table}

% modularity clustering
\begin{table}[p]
\centering
\caption{Computational results for \texttt{football}}
\bigskip
\mycsvtable{analysisCSVs/football_analysis.csv}
\label{tab:football}
\\[10pt]
The structure of the table is as for \cref{tab:cetacea}.
\end{table}

\begin{table}[p]
\centering
\caption{Computational results for \texttt{adjnoun}}
\bigskip
\mycsvtable{analysisCSVs/adjnoun_analysis.csv}
\label{tab:adjnoun}
\\[10pt]
The structure of the table is as for \cref{tab:cetacea}.
\end{table}

\begin{table}[p]
\centering
\caption{Computational results for \texttt{polbooks}}
\bigskip
\mycsvtable{analysisCSVs/polbooks_analysis.csv}
\label{tab:polbooks}
\\[10pt]
The structure of the table is as for \cref{tab:cetacea}.
\end{table}

\begin{table}[p]
\centering
\caption{Computational results for \texttt{lesmis}}
\bigskip
\mycsvtable{analysisCSVs/lesmis_analysis.csv}
\label{tab:lesmis}
\\[10pt]
The structure of the table is as for \cref{tab:cetacea}.
\end{table}

\begin{table}[p]
\centering
\caption{Computational results for \texttt{dolphins}}
\bigskip
\mycsvtable{analysisCSVs/dolphins_analysis.csv}
\label{tab:dolphins}
\\[10pt]
The structure of the table is as for \cref{tab:cetacea}.
\end{table}

\begin{table}[p]
\centering
\caption{Computational results for \texttt{karate}}
\bigskip
\mycsvtable{analysisCSVs/karate_analysis.csv}
\label{tab:karate}
\\[10pt]
The structure of the table is as for \cref{tab:cetacea}.
\end{table}

% organoids ordere by size and then hardness
\begin{table}[p]
\centering
\caption{Computational results for \texttt{organoid\_160\_hard}}
\bigskip
\mycsvtable{analysisCSVs/organoid_160_hard_analysis.csv}
\label{tab:organoid_160_hard}
\\[10pt]
The structure of the table is as for \cref{tab:cetacea}.
\end{table}

\begin{table}[p]
\centering
\caption{Computational results for \texttt{organoid\_160\_medium}}
\bigskip
\mycsvtable{analysisCSVs/organoid_160_medium_analysis.csv}
\label{tab:organoid_160_medium}
\\[10pt]
The structure of the table is as for \cref{tab:cetacea}.
\end{table}

\begin{table}[p]
\centering
\caption{Computational results for \texttt{organoid\_160\_soft}}
\bigskip
\mycsvtable{analysisCSVs/organoid_160_soft_analysis.csv}
\label{tab:organoid_160_soft}
\\[10pt]
The structure of the table is as for \cref{tab:cetacea}.
\end{table}

\begin{table}[p]
\centering
\caption{Computational results for \texttt{organoid\_100\_hard}}
\bigskip
\mycsvtable{analysisCSVs/organoid_100_hard_analysis.csv}
\label{tab:organoid_100_hard}
\\[10pt]
The structure of the table is as for \cref{tab:cetacea}.
\end{table}

\begin{table}[p]
\centering
\caption{Computational results for \texttt{organoid\_100\_medium}}
\bigskip
\mycsvtable{analysisCSVs/organoid_100_medium_analysis.csv}
\label{tab:organoid_100_medium}
\\[10pt]
The structure of the table is as for \cref{tab:cetacea}.
\end{table}

\begin{table}[p]
\centering
\caption{Computational results for \texttt{organoid\_100\_soft}}
\bigskip
\mycsvtable{analysisCSVs/organoid_100_soft_analysis.csv}
\label{tab:organoid_100_soft}
\\[10pt]
The structure of the table is as for \cref{tab:cetacea}.
\end{table}

\begin{table}[p]
\centering
\caption{Computational results for \texttt{organoid\_80\_hard}}
\bigskip
\mycsvtable{analysisCSVs/organoid_80_hard_analysis.csv}
\label{tab:organoid_80_hard}
\\[10pt]
The structure of the table is as for \cref{tab:cetacea}.
\end{table}

\begin{table}[p]
\centering
\caption{Computational results for \texttt{organoid\_80\_medium}}
\bigskip
\mycsvtable{analysisCSVs/organoid_80_medium_analysis.csv}
\label{tab:organoid_80_medium}
\\[10pt]
The structure of the table is as for \cref{tab:cetacea}.
\end{table}

\begin{table}[p]
\centering
\caption{Computational results for \texttt{organoid\_80\_soft}}
\bigskip
\mycsvtable{analysisCSVs/organoid_80_soft_analysis.csv}
\label{tab:organoid_80_soft}
\\[10pt]
The structure of the table is as for \cref{tab:cetacea}.
\end{table}

\begin{table}[p]
\centering
\caption{Computational results for \texttt{organoid\_40\_hard}}
\bigskip
\mycsvtable{analysisCSVs/organoid_40_hard_analysis.csv}
\label{tab:organoid_40_hard}
\\[10pt]
The structure of the table is as for \cref{tab:cetacea}.
\end{table}

\begin{table}[p]
\centering
\caption{Computational results for \texttt{organoid\_40\_medium}}
\bigskip
\mycsvtable{analysisCSVs/organoid_40_medium_analysis.csv}
\label{tab:organoid_40_medium}
\\[10pt]
The structure of the table is as for \cref{tab:cetacea}.
\end{table}

\begin{table}[p]
\centering
\caption{Computational results for \texttt{organoid\_40\_soft}}
\bigskip
\mycsvtable{analysisCSVs/organoid_40_soft_analysis.csv}
\label{tab:organoid_40_soft}
\\[10pt]
The structure of the table is as for \cref{tab:cetacea}.
\end{table}

% random binary
\begin{table}[p]
\centering
\caption{Computational results for \texttt{random\_binary\_s25}}
\bigskip
\mycsvtable{analysisCSVs/random_binary_s25_analysis.csv}
\label{tab:random_binary_s25}
\\[10pt]
The structure of the table is as for \cref{tab:cetacea}.
\end{table}

% random continuous uniform
\begin{table}[p]
\centering
\caption{Computational results for \texttt{random\_uniform\_s25\_-1\_1}}
\bigskip
\mycsvtable{analysisCSVs/random_uniform_s25_-1_1_analysis.csv}
\label{tab:random_uniform_s25_-1_1}
\\[10pt]
The structure of the table is as for \cref{tab:cetacea}.
\end{table}

\begin{table}[p]
\centering
\caption{Computational results for \texttt{random\_uniform\_s25\_-100\_100}}
\bigskip
\mycsvtable{analysisCSVs/random_uniform_s25_-100_100_analysis.csv}
\label{tab:random_uniform_s25_-100_100}
\\[10pt]
The structure of the table is as for \cref{tab:cetacea}.
\end{table}

\begin{table}[p]
\centering
\caption{Computational results for \texttt{random\_uniform\_s25\_-10\_100}}
\bigskip
\mycsvtable{analysisCSVs/random_uniform_s25_-10_100_analysis.csv}
\label{tab:random_uniform_s25_-10_100}
\\[10pt]
The structure of the table is as for \cref{tab:cetacea}.
\end{table}

% random normal mu = 0
\begin{table}[p]
\centering
\caption{Computational results for \texttt{random\_normal\_s25\_0\_0.5}}
\bigskip
\mycsvtable{analysisCSVs/random_normal_s25_0_0.5_analysis.csv}
\label{tab:random_normal_s25_0_0.5}
\\[10pt]
The structure of the table is as for \cref{tab:cetacea}.
\end{table}

\begin{table}[p]
\centering
\caption{Computational results for \texttt{random\_normal\_s25\_0\_1}}
\bigskip
\mycsvtable{analysisCSVs/random_normal_s25_0_1_analysis.csv}
\label{tab:random_normal_s25_0_1}
\\[10pt]
The structure of the table is as for \cref{tab:cetacea}.
\end{table}

\begin{table}[p]
\centering
\caption{Computational results for \texttt{random\_normal\_s25\_0\_2}}
\bigskip
\mycsvtable{analysisCSVs/random_normal_s25_0_2_analysis.csv}
\label{tab:random_normal_s25_0_2}
\\[10pt]
The structure of the table is as for \cref{tab:cetacea}.
\end{table}

% random normal mu = 0.5
\begin{table}[p]
\centering
\caption{Computational results for \texttt{random\_normal\_s25\_0.5\_0.5}}
\bigskip
\mycsvtable{analysisCSVs/random_normal_s25_0.5_0.5_analysis.csv}
\label{tab:random_normal_s25_0.5_0.5}
\\[10pt]
The structure of the table is as for \cref{tab:cetacea}.
\end{table}

\begin{table}[p]
\centering
\caption{Computational results for \texttt{random\_normal\_s25\_0.5\_1}}
\bigskip
\mycsvtable{analysisCSVs/random_normal_s25_0.5_1_analysis.csv}
\label{tab:random_normal_s25_0.5_1}
\\[10pt]
The structure of the table is as for \cref{tab:cetacea}.
\end{table}

\begin{table}[p]
\centering
\caption{Computational results for \texttt{random\_normal\_s25\_0.5\_2}}
\bigskip
\mycsvtable{analysisCSVs/random_normal_s25_0.5_2_analysis.csv}
\label{tab:random_normal_s25_0.5_2}
\\[10pt]
The structure of the table is as for \cref{tab:cetacea}.
\end{table}

% random normal mu = 2
\begin{table}[p]
\centering
\caption{Computational results for \texttt{random\_normal\_s25\_2\_0.5}}
\bigskip
\mycsvtable{analysisCSVs/random_normal_s25_2_0.5_analysis.csv}
\label{tab:random_normal_s25_2_0.5}
\\[10pt]
The structure of the table is as for \cref{tab:cetacea}.
\end{table}

\begin{table}[p]
\centering
\caption{Computational results for \texttt{random\_normal\_s25\_2\_1}}
\bigskip
\mycsvtable{analysisCSVs/random_normal_s25_2_1_analysis.csv}
\label{tab:random_normal_s25_2_1}
\\[10pt]
The structure of the table is as for \cref{tab:cetacea}.
\end{table}

\begin{table}[p]
\centering
\caption{Computational results for \texttt{random\_normal\_s25\_2\_2}}
\bigskip
\mycsvtable{analysisCSVs/random_normal_s25_2_2_analysis.csv}
\label{tab:random_normal_s25_2_2}
\\[10pt]
The structure of the table is as for \cref{tab:cetacea}.
\end{table}

% random binary
\begin{table}[p]
\centering
\caption{Computational results for \texttt{random\_binary\_s50}}
\bigskip
\mycsvtable{analysisCSVs/random_binary_s50_analysis.csv}
\label{tab:random_binary_s50}
\\[10pt]
The structure of the table is as for \cref{tab:cetacea}.
\end{table}

% random continuous uniform
\begin{table}[p]
\centering
\caption{Computational results for \texttt{random\_uniform\_s50\_-1\_1}}
\bigskip
\mycsvtable{analysisCSVs/random_uniform_s50_-1_1_analysis.csv}
\label{tab:random_uniform_s50_-1_1}
\\[10pt]
The structure of the table is as for \cref{tab:cetacea}.
\end{table}

\begin{table}[p]
\centering
\caption{Computational results for \texttt{random\_uniform\_s50\_-100\_100}}
\bigskip
\mycsvtable{analysisCSVs/random_uniform_s50_-100_100_analysis.csv}
\label{tab:random_uniform_s50_-100_100}
\\[10pt]
The structure of the table is as for \cref{tab:cetacea}.
\end{table}

\begin{table}[p]
\centering
\caption{Computational results for \texttt{random\_uniform\_s50\_-10\_100}}
\bigskip
\mycsvtable{analysisCSVs/random_uniform_s50_-10_100_analysis.csv}
\label{tab:random_uniform_s50_-10_100}
\\[10pt]
The structure of the table is as for \cref{tab:cetacea}.
\end{table}

% random normal mu = 0
\begin{table}[p]
\centering
\caption{Computational results for \texttt{random\_normal\_s50\_0\_0.5}}
\bigskip
\mycsvtable{analysisCSVs/random_normal_s50_0_0.5_analysis.csv}
\label{tab:random_normal_s50_0_0.5}
\\[10pt]
The structure of the table is as for \cref{tab:cetacea}.
\end{table}

\begin{table}[p]
\centering
\caption{Computational results for \texttt{random\_normal\_s50\_0\_1}}
\bigskip
\mycsvtable{analysisCSVs/random_normal_s50_0_1_analysis.csv}
\label{tab:random_normal_s50_0_1}
\\[10pt]
The structure of the table is as for \cref{tab:cetacea}.
\end{table}

\begin{table}[p]
\centering
\caption{Computational results for \texttt{random\_normal\_s50\_0\_2}}
\bigskip
\mycsvtable{analysisCSVs/random_normal_s50_0_2_analysis.csv}
\label{tab:random_normal_s50_0_2}
\\[10pt]
The structure of the table is as for \cref{tab:cetacea}.
\end{table}

% random normal mu = 0.5
\begin{table}[p]
\centering
\caption{Computational results for \texttt{random\_normal\_s50\_0.5\_0.5}}
\bigskip
\mycsvtable{analysisCSVs/random_normal_s50_0.5_0.5_analysis.csv}
\label{tab:random_normal_s50_0.5_0.5}
\\[10pt]
The structure of the table is as for \cref{tab:cetacea}.
\end{table}

\begin{table}[p]
\centering
\caption{Computational results for \texttt{random\_normal\_s50\_0.5\_1}}
\bigskip
\mycsvtable{analysisCSVs/random_normal_s50_0.5_1_analysis.csv}
\label{tab:random_normal_s50_0.5_1}
\\[10pt]
The structure of the table is as for \cref{tab:cetacea}.
\end{table}

\begin{table}[p]
\centering
\caption{Computational results for \texttt{random\_normal\_s50\_0.5\_2}}
\bigskip
\mycsvtable{analysisCSVs/random_normal_s50_0.5_2_analysis.csv}
\label{tab:random_normal_s50_0.5_2}
\\[10pt]
The structure of the table is as for \cref{tab:cetacea}.
\end{table}

% random normal mu = 2
\begin{table}[p]
\centering
\caption{Computational results for \texttt{random\_normal\_s50\_2\_0.5}}
\bigskip
\mycsvtable{analysisCSVs/random_normal_s50_2_0.5_analysis.csv}
\label{tab:random_normal_s50_2_0.5}
\\[10pt]
The structure of the table is as for \cref{tab:cetacea}.
\end{table}

\begin{table}[p]
\centering
\caption{Computational results for \texttt{random\_normal\_s50\_2\_1}}
\bigskip
\mycsvtable{analysisCSVs/random_normal_s50_2_1_analysis.csv}
\label{tab:random_normal_s50_2_1}
\\[10pt]
The structure of the table is as for \cref{tab:cetacea}.
\end{table}

\begin{table}[p]
\centering
\caption{Computational results for \texttt{random\_normal\_s50\_2\_2}}
\bigskip
\mycsvtable{analysisCSVs/random_normal_s50_2_2_analysis.csv}
\label{tab:random_normal_s50_2_2}
\\[10pt]
The structure of the table is as for \cref{tab:cetacea}.
\end{table}

\end{landscape}
