\section{Prior and Related Work}\label{sec:related}



Many variations of the problem are well studied, e.g. restricting the cardinality or the number (or both) of the obtained partitions.\todo{add references to original papers}
Partitioning possibly incomplete graphs is also studied under the name “\textsc{Multicut} problem”.
Here, one typically thinks of the edges that are ‘cut’, \ie the edges between blocks of the partitiong instead of within a block.
Hence the characteristic vectors and the inequalities for the \textsc{Multicut} problem are obtained by replacing a variable $x$ from the \CP problem with $(1-x)$.
Therefore, these two problems are equivalent for complete graphs in the sense that an optimal solution to one of them also yields an optimal solution to the other.
