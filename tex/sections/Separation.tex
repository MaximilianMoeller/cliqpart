\section{Separation}\label{sec:separation}
In this section we will describe the classes of inequalities that we separated as well as the algorithms used to do so.
Most of the separation routines are configurable in one way or another.
However, we will only explain the core algorithm in this section and leave the configuration parameters for \cref{subsec:run_configs}.

\subsection{Triangle inequalities}\label{subsec:triangle_separator}
In any complete graph $K_{n}$ there are only $3n(n-1)(n-2) = \mathcal{O}(n^{3})$ triangle inequalities.
We found that complete enumeration runs in acceptable time even for the biggest instances we tested.

\subsection{\texorpdfstring{$2$-partition}{2-partition} inequalities}\label{subsec:two_partition_separator}
The $2$-partition inequalities first described by \cite{grotschelFacetsCliquePartitioning1990} are a generalization of triangle inequalities.
\begin{definition}[$2$-partition inequalities, Section 4 of \cite{grotschelFacetsCliquePartitioning1990}]\label{def:2-partition_inequality}
	Let $K_{n} = (V_{n}, E_{n})$ be a complete graph.
	For every two disjoint nonempty subsets $S$ and $T$ of $V_{n}$, the inequality
	\begin{equation}\label{eq:2-partition_inequality}
		\sum_{s \in S}^{} \sum_{t \in T}^{} x_{st} - \sum_{\substack{s, s' \in S \\ s \neq s'}}^{} x_{ss'} - \sum_{\substack{t, t' \in T \\ t \neq t'}}^{} x_{tt'} \leq \func{min}( \lvert S \rvert, \lvert T \rvert )
	\end{equation}
	is called a $2$\textit{-partition inequality}.
\end{definition}
\begin{theorem}[Theorem 4.1 of \cite{grotschelFacetsCliquePartitioning1990}]
	For every $n \geq 3$ and every two nonempty subsets $S, T$ of $V_{n}$, the corresponding $2$-partition inequality is valid for $\mathscr{P}_{n}$.
	It defines a facet if and only if $\lvert  S \rvert \neq \lvert T \rvert$.
\end{theorem}
\cite{oostenCliquePartitioningProblem2001a} showed that separation of 2-partition inequalities is already \textsc{NP}-hard for any fixed $\lvert S \rvert$.
We therefore resorted to the heuristic separation routines described in \cite{grotschelFacetsCliquePartitioning1990}.

The 2-partition inequalities are even further generalized by the \textit{clique-web inequalities} presented in \cite{dezaCliqueWebFacetsMulticut1992}.
This broader class does also contain other polynomial-time-seperable subclasses that we could not consider in the time available for the project.

\subsection{Two-chorded odd cycle inequalities}\label{subsec:two_chorded_separator}
\begin{definition}[Two-chorded cycle inequalities, Section 5 of \cite{grotschelFacetsCliquePartitioning1990}]\label{def:2-chorded_cycle_inequality}
	Let $K_{n} = (V_{n}, E_{n})$ be a complete graph, let $5 \leq k \leq n$, and let $v: \Z_{k} \to V_{n}$ be injective.
	The \textit{two-chorded cycle inequality} with respect to $v$ is defined as
	\begin{equation}\label{eq:two-chorded_cycle_inequality}
		\sum_{i \in \Z_{k}}^{} x_{v_{i}v_{i+1}} - x_{v_{i}v_{i+2}} \leq \left\lfloor \frac{1}{2} k \right\rfloor
	\end{equation}
\end{definition}
\begin{theorem}[Theorem 5.1 of \cite{grotschelFacetsCliquePartitioning1990}]
	Let $K_{n}$ be a complete graph, let $5 \leq k \leq n$, and let $v: \Z_{k} \to V_{n}$ be injective.
	Then the two-chorded cycle inequality with respect to $v$ is valid for $\mathscr{P}_{n}$.
	It defines a facet if and only if $k$ is odd.
\end{theorem}

The naive attempt of separating the class of two-chorded odd cycle inequalities by enumeration quickly breaks down,
as there are 
\[
	\sum_{k=5 \text{ odd}}^{n} \binom{n}{k} \frac{k!}{2k} = \sum_{k = 5 \text{ odd}}^{n} \frac{n!}{(n-k)!2k}
\]
distinct two-chorded odd cycle inequalities for $K_{n}$.
\begin{proof}
	Analogous to the Proof of Proposition 20 in \cite{andresPolyhedralStudyLifted2022}.
\end{proof}

Independently, the class was shown to be seperable in polynomial time by \cite{mullerPartialOrderPolytope1996} and \cite{capraraChvatalGomoryCuts1996}.
We will focus on the technique used by \cite{mullerPartialOrderPolytope1996}.
Their paper is mainly concerned with the partial order polytope of a directed graph.
The algorithm described for separating \textit{odd closed walk} inequalities can be adapted to to separate a class of valid inequalities for the \CP polytope, that contains the class of two-chorded odd cycle inequalities.

In this paper, we are only interested in the class of two-chorded inequalities (and not this broader class) and will therefore repeat the argument given, but slightly adapted for this narrower class.

Consider \cref{fig:support_graph_two-chorded_odd_cycle_inequality}.
We can obtain the left-hand side of the associated inequality by following the closed walk
\[
\mathrlap{\overbrace{\phantom{v_{0} v_{1} v_{2} v_{3} v_{4} v_{5} v_{6} v_{0}}}^{\text{cycle}}}
v_{0} v_{1} v_{2} v_{3} v_{4} v_{5} v_{6}
\mathrlap{\underbrace{\phantom{v_{0} v_{2} v_{4} v_{6} v_{1} v_{3} v_{5} v_{0}}}_{\text{two-chords}}}
v_{0} v_{2} v_{4} v_{6} v_{1} v_{3} v_{5} v_{0}
\]
and adding up the corresponding entries of the solution vector $x^{*}$ multiplied by $\pm 1$, respectively.
As long as we only traverse odd cycles and their two-chords (and not even ones) this will always produce the left-hand side of a two-chorded odd cycle inequality and, crucially, we do not need to know the $k$ in advance.
This suggests that we could try find some ‘special walk’ in an appropriate auxiliary graph $H$ where the structure of the graph prohibits us from ever taking walks along even cycles.

\begin{figure}
	\centering
	\begin{tikzpicture}
        \def \n{6};
        \def \radius{1.5cm};
        \def \r{0.3cm};
        \foreach \i in {0,...,\n}{
			\node[vertex] (\i) at ({\i/(\n+1)*360+90}:\radius) {};
			\node at ({\i/(\n+1)*360+90}:\radius+\r) {$v_{\i}$};
        }
        \foreach \i/\j in {0/1, 1/2, 2/3, 3/4, 4/5, 5/6, 6/0} {
			\draw[line width=0.2mm] (\i) -- (\j);
        }
        \foreach \i/\j in {0/2, 1/3, 2/4, 3/5, 4/6, 5/0, 6/1} {
			\draw[line width=0.2mm, dashed] (\i) -- (\j);
        }
	\end{tikzpicture}
	\caption[Support graph of a two-chorded odd cycle inequality]{Support graph of a two-chorded odd cycle inequality for $k$ = 7.
	Solid/Dashed lines represent variables with coefficient $+ / - 1$, respectively.}
	\label{fig:support_graph_two-chorded_odd_cycle_inequality}
\end{figure}

So restricting us to odd cycles, we can rewrite the right-hand side of \cref{eq:two-chorded_cycle_inequality}:
\begin{equation*}
	\sum_{i \in \Z_k} x_{v_i v_{i+1}} - x_{v_i v_{i+2}} \leq \frac{1}{2} (k-1)
\end{equation*}
By subtracting $\frac{1}{2}k$ from both sides and after rearranging we obtain:
\begin{equation}\label{eq:two-chorded_graph_weights}
	\sum_{i \in \Z_k} \left( \frac{1}{2} -x_{v_i v_{i+1}} + x_{v_i v_{i+2}} \right) \geq \frac{1}{2}
\end{equation}
Now this inequality is independent of $k$ when we use the special graph $H$ described earlier to find the left-hand side.
Furthermore, adding some ‘weights’ and checking whether they are samller than some constant strongly hints at the use of a shortest-path algorithm.

We define the (weighted, directed, simple) axiliary graph $H = (V_{H}, E_{H})$ as follows:
For every (directed) pair $(i,j) \in V_{n}^{2}$ with $i \neq j$, we add the four nodes $u_{1}^{i,j}, u_{2}^{i,j}, \hat{u}_{1}^{i,j}$ and $\hat{u}_{2}^{i,j}$ to $V_{H}$ 
and the two arcs $(u_{1}^{i,j}, u_{2}^{i,j})$ and $(\hat{u}_{1}^{i,j}, \hat{u}_{2}^{i,j})$ to $E_{H}$.
We will call four such nodes a \textit{gadget} and the two edges inside a gadget \textit{inner-gadget edges}.
Both inner-gadget edges get the weight $\frac{1}{2} - x_{ij}$.
Furthermore for every (directed) triple $(i,j,k) \in V_{n}^{3}$ with $i,j,k$ pairwise distinct, we add the two arcs $(u_{2}^{i,j}, \hat{u}_{1}^{j,k})$ and $(\hat{u}_{2}^{i,j}, u_{1}^{j,k})$ to $E_{H}$, called \textit{inter-gadget edges}.
Both of them are assigned the weight $x_{ik}$.
All edge weights are derived from \cref{eq:two-chorded_graph_weights}:
\begin{equation*}
	\sum_{i \in \Z_k} \biggl( \underbrace{ \frac{1}{2} -x_{v_i v_{i+1}} }_{\text{inner-gadget weights}} + \underbrace{x_{v_i v_{i+2}}}_{\text{inter-gadget weights}} \biggr) \geq \frac{1}{2}
\end{equation*}
%\cref{fig:aux_graph_K3} shows an example of the auxiliary graph for $K_{3}$.
%Note that $H$ connected for any $K_{n}$ with $n \geq 4$, but showing such a graph in full detail is rather confusing.

We use the special node types $u$ and $\hat{u}$ to distinguish between even and odd cycles.
In fact, $H$ can be thought of a special kind of bipartide graph divided into ${U = \big\{ u_{1}^{i,j}, u_{2}^{i,j} \mid 0 \leq i,j \leq n, i \neq j \big\}}$ and $\hat{U} = \big\{ \hat{u}_{1}^{i,j}, \hat{u}_{2}^{i,j} \mid 0 \leq i,j \leq n, i \neq j \big\}$.
This is obviously not a bipartition in the standart sense, because of the inner-gadget edges $(u_{1}^{i,j}, u_{2}^{i,j})$ and $(\hat{u}_{1}^{i,j}, \hat{u}_{2}^{i,j})$.
However, they can be disregarded, as they are only there to precisely obtain the left-hand side of the inequality.
The essential part is that we mark a special pair of nodes ($u_{1}^{i,j}$ and $\hat{u}_{1}^{i,j}$) in the two different parts of the bipartide graph which allows us to easily distinguish between odd and even cycles:
Every odd cycle $v_{0}v_{1} \ldots v_{k-1}v_{0}$ in $K_{n}$ where $k$ is odd induces a walk
\[
	u_{1}^{v_{0}, v_{1}}u_{2}^{v_{0}, v_{1}}\hat{u}_{1}^{v_{1}, v_{2}}\hat{u}_{2}^{v_{1}, v_{2}}
	\ldots
	\hat{u}_{1}^{v_{k-2}, v_{k-1}}\hat{u}_{2}^{v_{k-2}, v_{k-1}}u_{1}^{v_{k-1}, v_{0}}u_{2}^{v_{k-1}, v_{0}}\hat{u}_{1}^{v_{0}, v_{1}}
\]
in $H$, while every even cycle $v_{0}v_{1} \ldots v_{k-1}v_{0}$ with $k$ even induces a walk
\[
	u_{1}^{v_{0}, v_{1}}u_{2}^{v_{0}, v_{1}}\hat{u}_{1}^{v_{1}, v_{2}}\hat{u}_{2}^{v_{1}, v_{2}}
	\ldots
	u_{1}^{v_{k-2}, v_{k-1}}u_{2}^{v_{k-2}, v_{k-1}}\hat{u}_{1}^{v_{k-1}, v_{0}}\hat{u}_{2}^{v_{k-1}, v_{0}}u_{1}^{v_{0}, v_{1}}
.\]

Conversely, every $u_{1}^{i,j}\hat{u}_{1}^{i,j}$-walk in $H$ corresponds to an odd cycle in $K_{n}$ and the weight of this walk is precisely the left-hand side of \cref{eq:two-chorded_graph_weights}.
\cite{mullerPartialOrderPolytope1996} showed that $H$ does not contain any cycles with summed weight less than $0$ as long as the solution vector $x^{*}$ used in the construction satisfies the triangle inequalities.
Therefore, assuming that all triangle inequalities are satisfied, $x^{*}$ violates a two-chorded odd cycle inequality if and only if there exists a $u_{1}^{i,j}\hat{u}_{1}^{i,j}$-walk in $H$ with length less than $\frac{1}{2}$.
\cref{fig:walk_in_auxiliary_graph} shows such a walk in the auxiliary graph of $K_{7}$.


% Took so long to create, but sadly not helpful in explaining the subject :'(
\hide{
\begin{figure}[p]
	\centering
	\begin{tikzpicture}
		\foreach \i/\j [count=\gc from 0] in {0/1, 1/2, 2/0, 0/2, 2/1, 1/0}
			\foreach \io in {1,2}
				\foreach \uv [count=\uvc from 0] in {u,v}{
					\pgfmathsetmacro{\desc}{ifthenelse(\uvc == 0, "$u_{\io}^{\i, \j}$", "$\hat{u}_{\io}^{\i, \j}$")}
					\node[vertex]
					(\uv\io\i\j) 
					at ({Mod(\gc,3)*3.5 - ifthenelse(\io == 2, 0, 1.5)}, {ifthenelse(\gc < 3, 0, 1)*2.75 - \uvc*1.5})
					{\desc};
			}
		\foreach \i/\j in {0/1, 0/2, 1/0, 1/2, 2/0, 2/1}{
			\draw[-latex] (u1\i\j) edge[line width=0.75pt] (u2\i\j);
			\draw[-latex] (v1\i\j) edge[line width=0.75pt] (v2\i\j);
		}
		\foreach \i/\j/\k [count=\cc from 1] in {0/1/2, 1/2/0, 2/0/1, 0/2/1, 2/1/0, 1/0/2}{
			\pgfmathsetmacro{\outdiruv}{ifthenelse(Mod(\cc,3) == 0, "south west", "east"}
			\pgfmathsetmacro{\outdirvu}{ifthenelse(Mod(\cc,3) == 0, "north west", "east"}
			\pgfmathsetmacro{\indiruv}{ifthenelse(Mod(\cc,3) == 0, "north east", "west"}
			\pgfmathsetmacro{\indirvu}{ifthenelse(Mod(\cc,3) == 0, "south east", "west"}
			\pgfmathsetmacro{\looseness}{ifthenelse(Mod(\cc,3) == 0, 0.2, 1}
			\draw[-latex] (u2\i\j) edge[line width=0.75pt, dashed, out=\outdiruv, in=\indiruv, looseness=\looseness] (v1\j\k);
			\draw[-latex] (v2\i\j) edge[line width=0.75pt, dashed, out=\outdirvu, in=\indirvu, looseness=\looseness] (u1\j\k);
		}
	\end{tikzpicture}
	\caption[Auxiliary graph $H$ for $K_3$]{The auxiliary graph for the complete graph $K_{3}$.
	Inner-gadget edges are shown as solid lines and inter-gadget edges are shown as dashed lines.}
	\label{fig:aux_graph_K3}
\end{figure}
}

\begin{figure}[H]
	\centering
	\resizebox{0.95\textwidth}{!}{
			\begin{tikzpicture}
        \def \n{6};
		\foreach \i in {0, ..., \n}
			\foreach \j in {0, ..., \n}{

			\ifthenelse{\i = \j}{}{

			\foreach \io in {1,2}
				\foreach \uv [count=\uvc from 0] in {u,v}{
					\pgfmathsetmacro{\desc}{ifthenelse(\uvc == 0, "$u_{\io}^{\i, \j}$", "$\hat{u}_{\io}^{\i, \j}$")}
					\node[vertex]
					(\uv\io\i\j) 
					at ({\i*3 + \uvc*1.25}, -{\j*3 + ifthenelse(\io == 2, 0, 1.25)}) %{};
					{\desc};
				}
			\draw[-latex] (u1\i\j) -- (u2\i\j);
			\draw[-latex] (v1\i\j) -- (v2\i\j);
			}

			}
		\foreach \i/\j/\k [count=\counter from 0] in {0/1/2, 1/2/3, 2/3/4, 3/4/5, 4/5/6, 5/6/0, 6/0/1}{
			\pgfmathsetmacro{\startnode}{ifthenelse(Mod(\counter,2) == 0, "u", "v")}
			\pgfmathsetmacro{\endnode}{ifthenelse(Mod(\counter,2) == 0, "v", "u")}
			\pgfmathsetmacro{\indir}{ifthenelse(\j == 6, "west", "north")}
			\pgfmathsetmacro{\outdir}{ifthenelse(\j == 6, "east", "south")}
			\pgfmathsetmacro{\looseness}{ifthenelse(\j == 0, 0.25, ifthenelse(\j == 6, 0.25, 0.5))}
			\draw[-latex] (\startnode2\i\j) edge[line width=0.75pt, color=cdblue, out=\outdir, in=\indir, looseness=\looseness] (\endnode1\j\k);
		}
		\foreach \i/\j [count=\counter from 0] in {0/1, 1/2, 2/3, 3/4, 4/5, 5/6, 6/0}{
			\pgfmathsetmacro{\uv}{ifthenelse(Mod(\counter,2) == 0, "u", "v")}
			\draw[-latex] (\uv1\i\j) edge[line width=0.75pt, color=cdblue] (\uv2\i\j);

		}
	\end{tikzpicture}}
	\caption[Walk in $H$]{The auxiliary graph $H$ for $K_{7}$.
	The $u_{1}^{0,1}\hat{u}_{1}^{0,1}$-walk corrsponding to the inequality in \cref{fig:support_graph_two-chorded_odd_cycle_inequality} is shown in blue.
	All other inter-gadget weights are left out for clarity.}
	\label{fig:walk_in_auxiliary_graph}
\end{figure}

In practise, the choice of the shortest path algorithm is important.
For $n \in \N$ the auxiliary graph $H$ for $K_{n}$ has the following dimensions:
\begin{align*}
	\lvert V_{H} \rvert &= 4 n(n-1) = 4n^{2} - 4n \\
	\lvert E_{H} \rvert &= \underbrace{2 n(n-1)}_{\text{inner-gadget edges}} + \underbrace{2 n(n-1)(n-2)}_{\text{inter-gadget edges}} = 2n^{3} - 4n^{2} + 2n \\
\end{align*}
Since the graph might contain edges with negative costs, Dijkstra’s Algorithm (\cite{dijkstraNoteTwoProblems1959}) is not applicable.
We implemented a Bellman-Ford algorithm (\cite{bellmanRoutingProblem1958}, \cite{fordNetworkFlowTheory1956}), \ie, a single-source-single-target shortest path algorithm that runs in time $\mathcal{O}(\lvert V_{H} \rvert \cdot \lvert E_{H} \rvert) = \mathcal{O}(n^{5})$.
We need to run the algorithm for every $u_{1}^{i,j}\hat{u}_{1}^{i,j}$-pair in the auxiliary graph, \ie, $\mathcal{O}(n^{2})$ many times resulting in an overall run-time of $\mathcal{O}(n^{7})$.
While a Floyd-Warshall algorithm (\cite{floydAlgorithm97Shortest1962}, \cite{warshallTheoremBooleanMatrices1962}) would take only $\mathcal{O}(\lvert V \rvert^3) = \mathcal{O}(n^{6})$ many steps in total, there are some disadvantages that discouraged us from using it.
Firstly, the Bellman-Ford algorithm has space complexity of $\mathcal{O}(\lvert V \rvert) = \mathcal{O}(n^{2})$ whereas the space complexity of the Floyd-Warshall algorithm is of the order $\mathcal{O}(\lvert V \rvert^{2}) = \mathcal{O}(n^{4})$, which caused the algorithm to exceed 64GB of memory for our largest instances.
Secondly, we do not have to compute all $\mathcal{O}(n^{2})$ many Bellman-Ford iterations, if we simply choose not to.
We could, \eg, terminate after some number of violated constraints have been found, or if some time limit is exceeded.
In this case we would still be able to report on the violated constraints found so far and add them to the next iteration of the LP.
This is simply not possible for the Floyd-Warshall algorithm, which needs to be computed completely before any violated constraint can be found.


\subsection{Half-chorded odd cycle inequalities}\label{subsec:half_chorded_separator}

\begin{definition}[Half-chorded odd cycle inequalities, Definition 10 of \cite{andresPolyhedralStudyLifted2022}]\label{def:half-chorded}
	Let $5 \leq k \leq n$ with $k$ odd, let $d = \frac{k-1}{2}$, and let $v: \Z_{k} \to V_{n}$ be injective.
	The \textit{half-chorded odd cycle inequality} with respect to $v$ is defined as
	\begin{equation}\label{eq:half-chorded}
		\sum_{i \in \Z_{k}}^{} (x_{v_{i} v_{i+1}} - x_{v_{i} v_{i+d} }) \leq k - 3
	\end{equation}
\end{definition}
\begin{theorem}[Proposition 22 of \cite{andresPolyhedralStudyLifted2022}]
	Let $K_{n}$ be a complete graph, let $5 \leq k \leq n$, and let $v: \Z_{k} \to V_{n}$ be injective.
	Then the half-chorded cycle inequality with respect to $v$ is valid for $\mathscr{P}_{n}$.
	It defines a facet if and only if $k$ is odd.
\end{theorem}
Separation of the half-chorded odd cycle inequalities can be done analogously to the two-chorded odd cycle inequalities. 
Note that the half-chords $v_{i}v_{i+d}$ for $i \in \Z_{k}$ also form a cycle in $K_{n}$, and the edges $v_{i}v_{i+1}$ are the $2$-chords of that cycle.
We can therefore rewrite this equation as
\begin{equation*}
\sum_{i \in \Z_{k}}^{} (x_{w_{i}w_{i+2}} - x_{w_{i}w_{i+1}}) \leq k-3
\end{equation*}
where $w: \Z_{k} \to V_{n}, i \mapsto v_{i \cdot d}$.
By subtracting $k$ from both sides and multiplying by $-1$ we obtain:
\begin{equation*}
	\sum_{i \in \Z_{k}}^{} \Bigl( \underbrace{x_{w_{i}w_{i+1}}}_{\text{inner-gadget weight}} + \underbrace{1 -x_{w_{i}w_{i+2}}}_{\text{inter-gadged weight}} \Bigr) \geq 3.
\end{equation*}

The construction of $H$ is as above with the only difference being the value of the edge weights:
For every $(i,j) \in V^{2}$ with $i \neq j$ we set the weight of the two arcs $(u_{1}^{i,j},u_{2}^{i,j})$ and $(\hat{u}_{1}^{i,j},\hat{u}_{2}^{i,j})$ to be $x_{ij}$.
For every $(i,j,k) \in V^{3}$ with $i,j,k$ pairwise distinct, we set the weight of the inter-gadget edges $(u_{2}^{i,j},\hat{u}_{1}^{j,k})$ and $(\hat{u}_{2}^{i,j}, u_{1}^{j,k})$ to be $1 - x_{i,k}$.
Note that the weights in $H$ are all non-negative now, and hence we can use Dijkstra’s algorithm to check whether any $u_{1}^{i,j}\hat{u}_{1}^{i,j}$-walk has weight strictly smaller than $3$.
This reduces running time to $\mathcal{O}(\lvert E_{H} \rvert + \lvert V_{H} \rvert \cdot\func{log} \lvert V_{H} \rvert) = \mathcal{O}(n^{3})$.
As a further improvement, we can skip expanding vertices in the routine, whenever their current distance to the start vertex is already bigger than $3$, resulting in even faster average time.
