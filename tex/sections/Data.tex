\section{Data}\label{sec:data}
In this section we will describe the data we used to test the separation routines.
First we will present the instances available from the literature and then random instances we created.
Every instance will get an abbreviation for later reference that we will denote by “Instance name (\texttt{Abbreviation})”.
Furthermore, some of the instances can be parametrized in some way, \eg, the size of the instance, or the parameters of a random distribution.
We will include such parametrization parameters in the abbreviation.

\subsection{Real-world data}
\subsubsection{Cetacea (\texttt{Cet})}\label{subsubsec:data_cetacea}
We copied this instance from \cite{grotschelCuttingPlaneAlgorithm1989}, in order to sanity-check our implementation.
It is one of the easiest instances we tested, because an integral solution is already found after 2 iterations by just separating triangle inequalities.

\subsubsection{Organoid Clustering (\texttt{organoid\_[size]\_[difficulty]})}
In his master thesis, Jannik Presberger studied correlation clustering of organoids in light microscopy, where \CP showed up naturally as a the problem of clustering these organoids based on pairwise similarities.
The findings of this work are yet to be published, but we obtained one instance after private communication.
It consists of pairwise similarities of 160 organoids measured in an interval $(0,1)$, where a value of $1$ means that two organoids are identical and a value of $0$ means they are dissimilar.
A challenge in modelling the similarity between different organoids was mapping one organoids components to those of another organoid.
Their solution resulted in an asymmetric cost matrix, \ie for two organoids $i, j$ the similarity value $s(i,j)$ would not necessarily equal $s(j,i)$.
For creating a \CP instance from this cost matrix we followed them in choosing the smaller of the two possibilities for $w_{ij}$.
Note that these are maximization problems instead of the usual minimization.

Since these are all positive values, the resulting \CP instance would be trivial.
We therefore offset the values by subtracting a constant value $b$ from every one.
Jannik Presberger informed us that $b \in \left\{ 0.500, 0.550, 0.585 \right\}$ were adequate values for such an offset and provided us with the optimal values for each of the corresponding instances.
The running time of the ILP solver increases drastically as the offset approaches $b = 0.500$, which is to be expected, as this creates the most ‘conflicts’.
However, we have also been informed that the obtained clusterings tend to be more meaningful for $b$’s in the vicinity of $0.500$.
We have decided to call these instance \texttt{hard} ($b= 0.500$), \texttt{medium} ($b = 0.50$) and \texttt{soft} ($b = 0.585$), respectively.
Furtermore, we have created subsets of these instances by just considering the first \texttt{size} organoids in the cost matrix.



\subsubsection{Modularity Clustering}
Janniks data is from \cite{kappesComparativeStudyModern2015}, but that is not the original source...
\todo{Write this section and add original source}


\subsection{Random Data}\label{subsec:random_data}
We also created \CP instances randomly and for different sizes.
We will abbreviate all random instances starting with an \texttt{r\_} followed by the name and size of the instance,
\eg, \texttt{r\_normal\_200\_[further parameters]} is an instance of $K_{200}$ with normally distributed edge weights according to further parameters.

\subsubsection{Binary data (\texttt{r\_binary\_[size])}}\label{subsubsec:data_random_binary}
The edge weights of the input graph are sampled from the set $\left\{ -1, 1 \right\}$ with equal probability.
We created these instances with the intetion of being computationally ‘hard’.
That is because choosing between rewarding and penalizing edges randomly with the same numeric incentive creates many ‘conflicts’.

\subsubsection{Uniform distribution (\texttt{r\_uniform\_[size]\_[lb]\_[ub]})}\label{subsubsec:data_random_uniform}
In this instance type the edge weights are sampled from a continuous uniform distribution with interval boundaries \texttt{lb} below and \texttt{ub} above and with $\text{\texttt{lb}} < \text{\texttt{ub}}$.
We suspected these instances to be easier to solve than the binary data, especially when using asymetric boundaries, \eg \texttt{r\_uniform\_[size]\_-10\_+100}, because in an instance like this, ‘conflicts’ are often already solved by the numeric value of the ‘conflicting’ edges.

\subsubsection{Normal distribution (\texttt{r\_normal\_[size]\_[mu]\_[sigma]})}\label{subsubsec:data_random_normal}
Real-world data can often be described with normal distributions, which raises the question how applicable the separation routines are in practise.
The edge weights of these instances are sampled from a normal distribution with mean \texttt{mu} and standard deviation \texttt{sigma}.
We expected that a non-zero mean results in ‘easier’ instances, as many of the edges could be resolved trivially.
Conversely, we expected little to no influence of the standard deviation to the ‘hardness’ of the instance.
We have created instances for $(\text{\texttt{mu}}, \text{\texttt{sigma}}) \in \left\{ 0, 0.5, 2 \right\} \times \left\{ 0.5, 1, 2 \right\}$.
