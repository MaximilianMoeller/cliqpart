\section{Separators}\label{sec:sep}
This section is concerned with the separation routines (called \textit{separators} for short) for the classes of inequalities described in \cref{sec:cpp}.

\subsection{Triangle inequalities}\label{subsec:triangel_separator}

\subsection{\texorpdfstring{\ST}{[S:T]} inequalities}\label{subsec:st_separator}

\subsection{Two-chorded odd cycle inequalities}
The class of two-chorded odd cycle inequalities was shown to be seperable in polynomial time by \cite{mullerPartialOrderPolytope1996}.
Their paper is mainly concerned with the partial order polytope of a \textit{directed} graph.
However, the algorithm described there for separating \textit{odd closed walk} inequalities can be adapted to to separate a class of valid inequalities for the \CP polytope, that contains the class of two-chorded odd cycle inequalities.

We are however only interested in the class of two-chorded inequalities (and not this broader class) and will therefore repeat the argument given, but slightly adapted for this narrower class.

Since 2-chorded cycle inequalities are only facet defining for cycles of odd length, we can also rewrite \ref{eq:2-chords} as

\begin{equation*}
	\sum_{i \in \Z_k} \left( x_{v_i v_{i+1}} - x_{v_i v_{i+2}} \right) \leq \frac{1}{2} (k-1)
\end{equation*}

By subtracting $\frac{1}{2}k$ from both sides and after rearranging we obtain:

\begin{equation*}
	\sum_{i \in \Z_k} x_{v_i v_{i+1}} - \sum_{i \in \Z_k} \left(x_{v_i v_{i+2}} + \frac{1}{2} \right) \leq -\frac{1}{2}
\end{equation*}

\subsection{Half-chorded odd cycle inequalities}

\begin{definition}[Half-chorded odd cycle inequalities]\label{def:half-chorded}
Let $5 \leq k \leq n$ with $k$ odd, let $d = \frac{k-1}{2}$, and let $v: \Z_{k} \to \Z_{n}$ be injective.
The \textit{half-chorded odd cycle inequality} with respect to $v$ is defined as
\begin{equation}\label{eq:half-chorded}
	\sum_{i \in \Z_{k}}^{} (x_{v_{i} v_{i+1}} - x_{v_{i} v_{i+d} }) \leq k - 3
\end{equation}
\end{definition}
\begin{note}
	The half-chords $v_{i}v_{i+d}$ for $i \in \Z_{k}$ also form a cycle in $K_{n}$, and the edges $v_{i}v_{i+1}$ are the $2$-chords of that cycle.
\end{note}
We can therefore rewrite this equation as
\begin{equation*}
\sum_{i \in \Z_{k}}^{} (x_{w_{i}w_{i+2}} - x_{w_{i}w_{i+1}}) \leq k-3
\end{equation*}
where $w: \Z_{k} \to \Z_{n}, i \mapsto v_{i \cdot d}$.
By subtracting $k$ from both sides and multiplying by $-1$ we obtain:
\begin{equation*}
	\sum_{i \in \Z_{k}}^{} \left( -x_{w_{i}w_{i+2}} + x_{w_{i}w_{i+1}} + 1 \right) \geq 3.
\end{equation*}

Now for every tuple $(i,j) \in V^{2}$ we set the weights of the arcs $(u_{1}^{i,j}, u_{2}^{i,j})$ and $(v_{1}^{i,j}, v_{2}^{i,j})$ to $x_{ij}$.
And for every triple $(i,j,k) \in V^{3}$ we set the weight of the arcs $(u_{2}^{i,j}, v_{1}^{j,k})$ and $(v_{2}^{i,j}, u_{1}^{j,k})$ to $1-x_{ik}$.
Then, $x$ violates a half-chorded odd cycle inequality iff for any $(i,j) \in V^{2}$ there exists a $u_{1}^{i,j}v_{1}^{i,j}$-walk in $H$ with weighted lenght stricly less than 3.
